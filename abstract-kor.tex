\clearpage
\begin{flushleft}
    \Large{
        \textbf{국 문 초 록}\\
    }
\vspace{\baselineskip}
\end{flushleft}

\begin{centering}
    \LARGE{
        \textbf{논 문 제 목} \\
    }
\vspace{\baselineskip}
\end{centering}


\begin{flushright}
    이름 \\
    학과 \\
    일반대학원, 연세대학교
\end{flushright}

대한민국은 통일을 지향하며. 법률이 정하는 바에 의하여 대법관이 아닌 법관을 둘 수 있다. 법관은 탄핵 또는 금고 이상의 형의 선고에 의하지 아니하고는 파면되지 아니하며, 이에 의하여 민사상이나 형사상의 책임이 면제되지는 아니한다.

제안된 헌법개정안은 대통령이 20일 이상의 기간 이를 공고하여야 한다. 신체장애자 및 질병·노령 기타의 사유로 생활능력이 없는 국민은 법률이 정하는 바에 의하여 국가의 보호를 받는다. 감사원은 세입·세출의 결산을 매년 검사하여 대통령과 차년도국회에 그 결과를 보고하여야 한다. 체포·구속·압수 또는 수색을 할 때에는 적법한 절차에 따라 검사의 신청에 의하여 법관이 발부한 영장을 제시하여야 한다.

\blfootnote{핵심되는 말: 키워드1, 키워드2, 키워드3}
%\pagenumbering{gobble}  %remove page number on summary page
